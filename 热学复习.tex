    
    %!TEX TS-program = xelatex  
    %!TEX encoding = UTF-8 Unicode  
      
\documentclass[a4paper,12pt]{article}  
\usepackage[top=1.5in, bottom=1.5in, left=1.0in, right=1.0in]{geometry} 
\usepackage{longtable}
\usepackage{fancyhdr}
\usepackage{multirow}
\usepackage{setspace}
\usepackage{booktabs}
\usepackage{indentfirst}        
\usepackage{fontspec,xltxtra,xunicode}
\usepackage{float}
\usepackage{amsmath}

\usepackage[]{xeCJK}
\setCJKmainfont[BoldFont=STKaitiSC-Bold, ItalicFont=STHeitiSC-Light]{STSong}
\setCJKsansfont[BoldFont=STHeiti]{STXihei}
\setCJKmonofont{STFangsong}

%\defaultfontfeatures{Mapping=tex-text}  
%\setromanfont{SimSun} %设置中文字体  
\XeTeXlinebreaklocale “zh”  
\XeTeXlinebreakskip = 0pt plus 1pt minus 0.1pt %文章内中文自动换行  
      
%\newfontfamily{\H}{SimHei}  
%\newfontfamily{\E}{Arial}  %设定新的字体快捷命令 
\newcommand{\Hc}{\mathcal{H}}
\newcommand{\myp}[3]{(\cfrac{\partial{#1}}{\partial{#2}})_{#3}}

\title{热学期末复习}
\author{ssh}
\date{\today}
\begin{document}
\maketitle


\section{概念题}
\subsection{第三定律}
三种表述:
\begin{itemize}
\item 能斯特:不可能通过有限步骤使物体达到绝对零度
\item 能斯特定理:系统在等温过程中的熵变随温度趋于绝对零度而趋于0。\\$lim_{T \rightarrow 0}(\Delta S)_{T} = 0$
\item 系统的熵随绝对温度趋于0
\end{itemize}

\subsection{准静态}
进行的足够缓慢,以至于系统连续经过的每个中间态都可近似地看成平衡态的过程。
\subsection{可逆过程}
无摩擦的准静态过程是可逆过程。(系统能够回到初始状态且对外界无影响)
\subsection{焦-汤系数}
节流过程前后:$\mu = \myp{T}{P}{H}$








\section{配分函数的用法}
离散能级下的定义:$Z = \sum_i g_i e^{-\beta \varepsilon_i}$。\par
连续能谱,令$g(\varepsilon)$为能谱密度,则$Z = \int g(\varepsilon) e^{- \beta \varepsilon} d\varepsilon$。\par
回想求解一个统计分布问题,要在两个约束下使系统的微观态数$W$取极大值。这两个约束(粒子总数,总能量)
\begin{subequations}
\begin{align}
\sum_i a_i &= N \\
\sum_i a_i \varepsilon_i & = E
\end{align}
\end{subequations}
求解
\begin{equation}
\delta (lnW - \alpha \sum_i a_i - \beta \sum_i a_i \varepsilon_i) = 0
\end{equation}
$\alpha, \beta$是这么来的,通常$\beta = \frac{1}{k_B T}$。\par
MB下, $\alpha = - \cfrac{\mu}{k_B T}$
\begin{equation}
a_i = g_i e^{-\alpha - \beta \varepsilon_i}
\end{equation}
BE(玻色子)和FD(费米子)的分布为:
\begin{equation}
a_i = \cfrac{g_i}{e^{\alpha \pm \beta \varepsilon_i}}
\end{equation}
\\
\indent 回到$Z$的用法:
\begin{subequations}
\begin{align}
N &= Z \cdot e^{-\alpha} \\
E &= -N \cfrac{\partial ln Z}{\partial \beta} \\
G &= -N \frac{1}{\beta} lnZ \\
P &= \cfrac{\partial G}{\partial V} \\ 
dS &= \frac{1}{T}(dU + PdV) = k_B \beta (N d(\cfrac{\partial ln Z}{\partial \beta}) + \frac{1}{\beta} N \cfrac{\partial ln Z}{\partial V} dV) \\
\end{align}
\end{subequations}
之后,$C_V$也可直接代入$\beta = \frac{1}{k_B T}$,$C_V = (\cfrac{\partial E}{\partial T})_V$。



\section{不同的热力学系统}
\subsection{热磁系统}
$(H, M, T)$系统,其中$M$是总磁矩:磁极化强度$\times$体积。\par
状态方程(居里定律):$M = C\cfrac{\Hc}{T}$。已知$C_{\Hc}(\Hc = 0) = \cfrac{b}{T^2}$。\par
由绝热功$dW = \mu_0 \Hc dM$,有第一定律$dU = dQ + \mu_0 \Hc dM$。$\Hc \sim P, M \sim V$。\par
一般来讲,实验控制的是励磁电流$\rightarrow$ $\Hc$。所以定义$H = U - \mu_0 \Hc M$, $dH = C_{\Hc}dT - \mu_0 M d\Hc$。若过程可逆,则有$dH = TdS - \mu_0 M d\Hc$,$dF = -SdT + \mu_0 \Hc dM$,$dG = -SdT - \mu_0 M d\Hc$。\par
因为$C_{\Hc}$是$T, \Hc$的函数,所以有
\begin{equation}
C_{\Hc} = C_{\Hc}(\Hc = 0) + \int_0^{\Hc} \myp{C_{\Hc}}{\Hc}{T} d\Hc 
\end{equation}
又由$dH$表达式知$C_{\Hc} = \myp{H}{T}{\Hc} = T\myp{S}{T}{\Hc}$。于是
\begin{equation}
\myp{C_{\Hc}}{\Hc}{T} = (\cfrac{T\partial{\myp{S}{T}{\Hc}}}{\partial{\Hc}})_{T} = T\myp{\myp{S}{\Hc}{T}}{T}{\Hc} 
\end{equation}
由$dG$表达式可得$\myp{S}{\Hc}{T} = \mu_0 \myp{M}{T}{\Hc} =  - \mu_0 C \cfrac{\Hc}{T^2}$\par
代回 $T\myp{\myp{S}{\Hc}{T}}{T}{\Hc}$
\begin{equation}
\myp{C_{\Hc}}{\Hc}{T} = T\myp{( - \mu_0 C \cfrac{\Hc}{T^2})}{T}{\Hc} = 2\mu_0 C \Hc \frac{1}{T^2} 
\end{equation}
代回(1)可积得:
\begin{equation}
C_{\Hc} = \cfrac{\mu_0 C \Hc^2 + b^2}{T^2}
\end{equation}

\indent 接下来就可以计算体系的熵:
\begin{subequations}
\begin{align*}
dS &= \myp{S}{T}{\Hc}dT + \myp{S}{\Hc}{T} d\Hc \\
   &= \cfrac{C_{\Hc}}{T}dT + \myp{\mu_0 M}{T}{\Hc} d\Hc \\
   &= \cfrac{\mu_0 C \Hc^2 + b^2}{T^3} - \cfrac{\mu_0 C \Hc}{T^2}d\Hc \\
   &= d[- \cfrac{b + \mu_0 C \Hc^2}{2 T^2}] \\
\end{align*}
\end{subequations}


\subsection{热辐射系统(光子气体)}
热辐射系统仍由$(P,V,T)$描述。状态方程:$p = \frac{1}{3}u(T)$。\par
基本微分方程:$dU = TdS - PdV$。\par
由$dF = -SdT - PdV$有$\myp{S}{V}{T} = \myp{P}{T}{V}$。所以
\begin{equation}
\myp{U}{V}{T} = T\myp{P}{T}{V} - P
\end{equation}
代入状态方程得:
\begin{equation}
u = \sigma T^4
\end{equation}
$\sigma$为积分常数。于是$P = \frac{1}{3}\sigma T^4$, $C_V = 4 \sigma T^3 V$。\par
计算体系的熵:
\begin{subequations}
\begin{align*}
dS &= \myp{S}{T}{V}dT + \myp{S}{V}{T} dV \\
   &= \cfrac{C_V}{T}dT + \myp{P}{T}{V}dV \\
   & = d[\frac{4}{3}\sigma T^3 V] \\
\end{align*}
\end{subequations}
\indent 光子气体的特性:$dG = -SdT + VdP = 0$。吉布斯自由能(化学势)守恒(为$0$)。

\subsection{膜理论}
一个薄膜有表面张力$F = 2 \alpha l$,$\alpha$一般与$T$成反比。于是膜系统由$(\alpha, A, T)$描述。\par
基本微分方程:$dU = TdS + \alpha dA$,$dF = -SdT + \alpha dA$,$dG = -SdT - Ad\alpha$。\par
设$U(T) = A \cdot u(T)$。广泛成立的式子是$\myp{U}{A}{T} = T\myp{\alpha}{T}{A} - \alpha$。于是有
\begin{equation}
u(T) = \alpha(T) - T\cfrac{d\alpha(T)}{T}
\end{equation}
计算得$C_A = -TA\cfrac{d^2\alpha}{dT^2}$。计算膜系统的熵:
\begin{subequations}
\begin{align*}
dS &= \myp{S}{T}{A}dT + \myp{S}{A}{T} dA \\
   &= \cfrac{C_A}{T}dT - \cfrac{d\alpha}{dT}dA \\
   &= d(- A \cfrac{d\alpha}{dT}) \\
\end{align*}
\end{subequations}
计算$dG = A \cfrac{d\alpha}{dT} dT - A d\alpha = 0$。膜系统与光子气体一样,吉布斯自由能守恒。


\section{相变}
\begin{subequations}
\begin{align*}
dG &= -S dT + V dP \\
d\mu &= -s dT + v dP \\
s &= \cfrac{\partial \mu}{\partial T} \\
v &= \cfrac{\partial \mu}{\partial P} \\
\end{align*}
\end{subequations}





























\end{document}
