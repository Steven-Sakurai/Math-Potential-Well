%!TEX TS-program = xelatex  
%!TEX encoding = UTF-8 Unicode  
      
\documentclass[a4paper, 11pt]{article}
\usepackage[top=1.5in, bottom=1.5in, left=1.0in, right=1.0in]{geometry} 
\usepackage{indentfirst}        
\usepackage{float}
\usepackage{amsmath}
\usepackage{graphicx}
\usepackage{braket}
\usepackage{setspace}
\usepackage{booktabs}        
\usepackage{fontspec,xltxtra,xunicode}

\usepackage[]{xeCJK}
\setCJKmainfont[BoldFont=STKaitiSC-Bold, ItalicFont=STHeitiSC-Light]{STSong}
\setCJKsansfont[BoldFont=STHeiti]{STXihei}
\setCJKmonofont{STFangsong}

%%%%%%%%%%%%%%%%%%%%%
%%%%%  MY MACRO  %%%%
\newcommand{\lap}[1]{\mathcal{L}[{#1}]}
\newcommand{\relap}[1]{\mathcal{L}^{-1}[{#1}]}


%%%%%%%%%%%%%%%%%%%%%

\title{数理期末复习}
\author{STEVN SHEN}
\date{\today}
\begin{document}
\maketitle

\section{Laplace Transform}
\begin{equation*}
\lap{f(t)} = \int_0 ^{+ \infty} f(t) e^{-pt} dt
\end{equation*}
\begin{equation*}
f(t)= \cfrac{1}{2\pi i}\int_{s - i\sigma} ^{s + i\sigma} F(p) e^{pt} dp
\end{equation*}

Convention: $\lap{f(t)} = F(p)$
\begin{enumerate}
\item $\lap{1} = \cfrac{1}{p}$, $\lap{e^{at}} = \cfrac{1}{p - a}$
\item $\lap{\delta(t)} = 1$, $\lap{\delta(t - t_0)} = e^{-pt_0}$
\item 
\begin{subequations}
\begin{align*}
\lap{e^{at}f(t)} &= F(p - a) \\
\lap{f(t - t_0)} &= e^{-pt_0}F(p)
\end{align*}
\end{subequations}
\item $\lap{f'(t)} = pF(p) - f(0)$
\item $\lap{\int_0^{t} f(\tau) d\tau} = \cfrac{F(p)}{p}$
\item $\lap{(-t)^n f(t)} = F^{(n)}(p)$
\item $\lap{\cfrac{f(t)}{t}} = \int_{p}^{\infty} F(q)dq$, \\
$p \rightarrow 0$, $\int_0^{\infty} \cfrac{f(t)}{t} = \int_0^{\infty} F(p)dp$
\item 若$F(p)$在无穷远点解析,即$F(p) = \sum_{n=1}^{\infty}c_n p^{-n}$\\
($F(p) \rightarrow 0$, $p \rightarrow \infty$) \\
逐项反演$f(t) = \sum_{n=0}^{\infty}\cfrac{c_{n+1}}{n!}t^n$
\item 由上条,$\lap{J_0(t)} = \cfrac{1}{\sqrt{p^2 + 1}}$\\
$\lap{J_0(2\sqrt{t})} = \cfrac{1}{p}e^{-p}$
\item 含参积分求导得:$\int_0^{\infty}e^{-t^2}cos2zt dt = \cfrac{\sqrt{\pi}}{2}e^{-z^2}$
\item $\lap{\cfrac{1}{\sqrt{\pi t}}\int_0^{\infty} f(\tau) e^{- \frac{\tau^2}{4t}} d\tau} = \cfrac{1}{\sqrt{p}}F(\sqrt{p})$
\item 由$\lap{\eta(t-a)} = \cfrac{1}{p}e^{-ap}$和上一条:\\
$\lap{erfc(\frac{a}{2\sqrt(t)})} = \cfrac{1}{p}e^{-a\sqrt{p}}$ \quad $erfc(x) = \cfrac{2}{\sqrt{\pi}} \int_x^{\infty} e^{-z^2} dz$
\item Convolution: $\lap{\int_0^t f_1(\tau) f_2(t-\tau) d\tau} = F_1(p)F_2(p)$
\end{enumerate}


\section{例题}
\subsection{积分变换}
\subsubsection{Fourier plus Laplace: 一维无界弦,$t=0$时,$u = \phi(x), u' = \psi(x)$}



\end{document}