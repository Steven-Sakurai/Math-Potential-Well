\documentclass[a4paper, 11pt]{article}
\usepackage{amsmath}
\usepackage{amssymb}
\usepackage{amsthm}
\usepackage{amsfonts}
\usepackage{braket}
\usepackage{indentfirst}
\title{\textbf{Notes on Quantum Physics}}
\author{Steven Shen}
\date{\today}
\begin{document}
\maketitle

\section{Experimental Facts}
\subsection{The Stern-Gerlach Experiment $1921 \sim 1922$}
For a heavy atom as a whole, magnetic moment $\mu$ is proportional to the electron spin $\overrightarrow{S}$, i.e. 
$\overrightarrow{\mu} = \frac{e\overrightarrow{S}}{m_e c}$. Place a magnetic field $\overrightarrow{B}$, the interaction energy yields $ - \overrightarrow{\mu} \cdot \overrightarrow{B}$. Therefore, if $B_z$ is not homogeneous, then $F_z = \mu_z \partial_z B_z$.\\
\indent We have a beam of silver atoms(47 electrons in total, with 46 of them having spherical symmetry, no net angular momentum, therefore the atom's angular momentum is sorely decided by the $47th$ electron), going through a $\overrightarrow{B}$ inhomogeneous in $B_z$. Then atoms will split in $z$ direction according to their spin. We only observe two distinct component on the other side, where $S_z = \pm \frac{\hbar}{2}$.\\
\indent The experiment suggests quantization of the electron spin angular momentum.
\subsection{Sequential SG Experiments}
$Oven \Longrightarrow SG\hat{z}$ (filtering $S_z -$) $\Longrightarrow SG\hat{z} \Longrightarrow$ $S_z +$ only \\
\indent $Oven \Longrightarrow SG\hat{z}$ (filtering $S_z -$) $\Longrightarrow SG\hat{x} \Longrightarrow$ $S_x+, S_x -$ \\
\indent $Oven \Longrightarrow SG\hat{z}$ (filtering $S_z -$) $\Longrightarrow SG\hat{z} \Longrightarrow SG\hat{z} \Longrightarrow S_z +, S_z -$ \\
\indent These results is similar to the polarization of light.\\
\indent Suppose $x \bot y, x^{\prime}=\frac{1}{\sqrt{2}}(x + y)$ \\                                                  $\Longrightarrow x-filter \Longrightarrow y-filter \Longrightarrow$ No light \\
$\Longrightarrow x-filter \Longrightarrow  x^{\prime}-filter \Longrightarrow y-filter \Longrightarrow$ $x_+, x_-$ \\
\indent The experiment suggests that we cannot determine $S_z$ and $S_z$ simultaneously. Previous information is destroyed by the new apparatus. It also suggests the superposition principle.\\
\indent Further we can use abstact vectors to represent the states in SG experiment on the basis of $\ket{S_z +}$ and $\ket{S_z -}$.
\newpage

\begin{subequations}
$\ket{S_x}, \ket{S_y}, \ket{S_z}$'s relations:
\begin{align}
	\ket{S_z +} &= \frac{1}{\sqrt{2}}(\ket{S_z +} + \ket{S_z -}),\\
	\ket{S_z -} &= \frac{1}{\sqrt{2}}(- \ket{S_z +} + \ket{S_z -}),\\
	\ket{S_y +} &= \frac{1}{\sqrt{2}}(\ket{S_z +} + i \ket{S_z -}),\\
	\ket{S_y -} &= \frac{1}{\sqrt{2}}(\ket{S_z +} - i \ket{S_z -})
\end{align}
\end{subequations}
\indent This example is clear to demonstrate the abstractness of vector space. Quantum-mechanical states are to be represented by vectors in an abstract complex vector space.\\

\subsection{Feynman's Remarks on electron's split}
See Quantum Field Theory in a Nutshell by A. Zee, Chap 1. Illustrate the motive of path integral. 

\section{Mathematics}

\subsection{Dirac, Ket \& Bra}
\subsubsection{Ket $\ket{\alpha}$ \& State Vector Space(Ket Space) $\mathcal{H}$}
We want it clear in the beginning. When referring to ket vectors, we're speaking of functions in a functional vector space, ex. functions expanded in fourier forms.
The dimension of the complex vector space is decided by the physics system's degree of freedom, ex. in the case of an electron's spin(upward \& downward), $dim = 2$.\\
\indent A physical state is represented by a \textbf{state vector} in a complex vector space, called \textbf{ket}, denoted by $\ket{\alpha}$. All information about that state is contained in the vector. Our postulation of the existence of vectors to represent states already suggests the superposition principle.\\
An observable, on the other hand, is represented by an operator which acts on the ket, on the left,\\
\begin{equation}
A \cdot (\ket{\alpha}) = A\ket{\alpha}
\end{equation}
\indent which yields another ket. If all kets of a system forms the space $\mathcal{H}$, an observable in that system is an operator on $\mathcal{H}$(operator defined as the same in linear algebra). Therefore, there should be eigenkets $\ket{\alpha'}$ of $A$,
\begin{equation}
A\ket{\alpha'} = \alpha' \ket{\alpha'}
\end{equation}
\indent Here $\alpha'$ is just a number. It's convention to `ket' an eigenvalue to stand for the corresponding eigenvector.
The physical state corresponding to an eigenket is called an eigenstate:
\begin{equation}
S_z\ket{S_z +} = \frac{\hbar}{2} \ket{S_z +}, S_z\ket{S_z -} = - \frac{\hbar}{2} \ket{S_z -}
\end{equation}
\indent Next, we consider a $N-dim$ vector space $\mathcal{H}$, spanned according to the $N$ eigenkets of observable $A$. Then $\forall \ket{\alpha} \in \mathcal{H}$, $\ket{\alpha} = \sum_{\alpha'} c_{\alpha'}\ket{\alpha'}$.

\subsubsection{Bra $\bra{\alpha}$, Dual Space(Bra Space) $\mathcal{H}^*$ and Inner Products}
Bra Space is dual to Ket Space. Corresponding to every $\ket{\alpha}$ in $\mathcal{H}$, there exists a bra in dual space $\mathcal{H}^*$, denoted by $\bra{\alpha}$.\\
\indent Inner product is defined as a mapping $i(*, *): \mathcal{H}\times\mathcal{H} \mapsto \mathbb{C}$. Think of $i(*, *)$ as a machine receiving two kets and returns a complex number. We all know how the inner product should be defined.\\
\indent Dual space is by now unclear. Yet with $i(*, *)$ we can deduce a dual space with a particular $i(*, *)$ defined. With $i(*, *)$ and $\forall f \in \mathcal{H}$, we have $i(f, *): \mathcal{H} \mapsto \mathbb{C}$, $i(f, *) \in \mathcal{H}^*$. Moreover, this is indeed a \textbf{one-to-one mapping}:
\begin{equation}
\forall \eta \in \mathcal{H}^*, \exists f_{\eta} \in \mathcal{H}, s.t. \forall g \in \mathcal{H}, \eta(g) = i(f_{\eta}, g)
\end{equation}
\indent Denote this mapping $v: \mathcal{H}^* \mapsto \mathcal{H}$, and denote $v^{-1}(\eta) = g_{\eta}$. \\
(Right now I can only understand dual space this way, remained to be uncovered.)\\
\indent Now we know a bra $\bra{\eta}$ acts on a ket $\ket{f}$ yields the $(g_{\eta}, f)$, which is called the inner product of $\bra{\eta}$ and $\ket{f}$ in physics, when mathematically this is inacurate.\\
\indent In physics, we often put a complex vector in $\mathcal{H}$ into a bra, ex. $\bra{\alpha}$, which actually means that $\bra{\alpha} = v^{-1}(\ket{\alpha})$. Though very confusing, we only need to remember finally $\bra{\eta}$ acts on $\ket{f}$ yields $(\ket{\eta}, \ket{f})$.

\subsubsection{Hermitian}
An observable's expectation value w.r.t. state $\ket{\Psi}$ is defined as:
\begin{equation}
	\braket{Q}_{\Psi} = \int\Psi^* \widehat{Q} \Psi dx = \braket{\Psi|\widehat{Q} \Psi}
\end{equation}
\indent An observable should be real. Therefore, the observable is represented by a Hermitian operator.










\end{document}
