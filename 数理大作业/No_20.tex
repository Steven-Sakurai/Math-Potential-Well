%!TEX TS-program = xelatex  
%!TEX encoding = UTF-8 Unicode  
\documentclass[12pt,a4paper]{paper}
\usepackage{float}
\usepackage{indentfirst}
\usepackage{geometry} 
\usepackage{enumitem} 
\usepackage{amsmath}  
\usepackage{braket}    
\usepackage{fontspec,xltxtra,xunicode}  

\usepackage[]{xeCJK}
\setCJKmainfont[BoldFont=STKaitiSC-Bold, ItalicFont=STHeitiSC-Light]{STSong}
\setCJKsansfont[BoldFont=STHeiti]{STXihei}
\setCJKmonofont{STFangsong}

%\defaultfontfeatures{Mapping=tex-text}  
%\setromanfont{SimSun} %设置中文字体  
\XeTeXlinebreaklocale “zh”  
\XeTeXlinebreakskip = 0pt plus 1pt minus 0.1pt %文章内中文自动换行  
      

\title{Black-Scholes Formula, a physicist's perspective}
\author{}
\date{\today}
\begin{document}
\maketitle

\section{(a)}
Rewritten Using $Brownian$ $Motion$: \\$ds(t) = \phi s(t) dt + \sigma s(t) dW(t)$, where $W(t)$ is a standard brownian motion.\\\indent To illustrate the relation between \textbf{Gaussian Noise} and \textbf{Brownian Motion}, consider when using $R(t)$, we're actually suggesting $s(t + \epsilon) = s(t) + \phi s(t) \epsilon + \sigma R \epsilon$. In this case, $R \sim \mathcal{N}(0, \frac{1}{\epsilon})$, therefore $R \epsilon \sim \mathcal{N}(0, \epsilon)$, which can be characterized as $W(t + \epsilon) - W(t)$. As $\epsilon \rightarrow 0$, $W(t + \epsilon) - W(t) \rightarrow dW(t)$. (It's really clearer to use $brownian$ $motion$ notation.) Brownian motion has the property that $dW(t)dW(t) = dt$, $dW(t) dt = 0$.\\
\indent The original statement can be rewritten as: \\$df(t, s(t)) = f_t dt + \frac{1}{2}\sigma^2s^2f_{ss}dt + \phi s f_s dt + \sigma s dW(t)$\\
\indent According to $Taylor$ $expansion$ $formula$, we can write
\begin{equation} 
df = f_t dt + f_s ds + \frac{1}{2}\{f_{tt} dt^2 + (f_{ts} + f_{st})dt ds(t) + f_{ss} ds(t)ds(t) \} + o(dt^2)
\end{equation} 

\begin{equation}
\Longrightarrow df = f_t dt + f_s ds + \frac{1}{2} f_{ss} ds(t)ds(t)
\end{equation}

\indent Considering $ds(t) = \phi s(t) dt + \sigma s(t) dW(t)$, $dW(t)dW(t) = dt$, $dW(t) dt = 0$, we have 
\begin{equation}
df = f_t dt + \frac{1}{2}\sigma^2s^2f_{ss}dt + \phi s f_s dt + \sigma s dW(t)
\end{equation}

\section{(b)}
$c = c(t, s(t))$, $d\Pi = c_t dt + c_s ds + \frac{1}{2}c_{ss} ds ds - c_s ds$.\\
$d\Pi = c_t dt + \frac{1}{2} c_{ss} \sigma^2 s^2 dt$

\section{(c)}
$d\Pi = r(c - c_s s)dt = c_t dt + \frac{1}{2} c_{ss} \sigma^2 s^2 dt$\\
$c_t + \frac{1}{2}\sigma^2 s^2 c_{ss} + rsc_s - rc = 0$

\section*{Another Way of Obtaining BS Formula}
According to non-arbitrage postulate, if at time $0$, $c(0, S(0)) = c_0$, then we should should be able to construct a portfolio $X(t)$(with $X(0) = c_0$) to replicate exactly this option. We should use the underlying stock $S(t)$ and the money market with interest rate $r$. Suppose we hold $\Delta(t)$ share of stock at time $t$, then it follows:
\begin{equation}
dX(t) = \Delta(t)dS(t) + r(X(t) - \Delta(t)S(t))dt = D_tdt + D_w dW(t)
\end{equation}
where $D_t = \Delta(t)S(t)(\phi - r) + r X(t)$, $D_w = \Delta(t)S(t)\sigma$.\\
\indent At the same time,
\begin{equation}
dc(t, S(t)) = c_t dt + c_s dS(t) + \frac{1}{2} c_{ss} dS(t)dS(t) = D_t' dt + D_w'dW(t)
\end{equation}
where $D_t' = c_t + c_s \phi S(t) + \frac{1}{2}\sigma^2 S(t)^2c_{ss}$, $D_w' = c_s \sigma S(t)$. \\
\indent Therefore, it should follows that $D_t = D_t', D_w = D_w'$. The second relation yields instantly that $\Delta(t) = c_s$, while the first would amount to the Black-Scholes formula:
\begin{equation}
c_t + \frac{1}{2}\sigma^2 s^2 c_{ss} + rsc_s - rc = 0
\end{equation}

\section{(d)}
Change variable $s = e^x$, we have $c_x = e^{-x}c_s$.\\
$c_t = rc - rsc_s - \frac{1}{2}\sigma^2 s^2 c_{ss} = (r - (r - \frac{1}{2}\sigma^2)\frac{\partial}{\partial x} - \frac{1}{2}\sigma^2 \frac{\partial^2}{\partial x^2}) c$.\\
\indent Therefore $H_{BS} = (1 - \frac{\partial}{\partial x})(r + \frac{1}{2} \sigma^2 \frac{\partial}{\partial x})$. 

\section{(e)}
\begin{equation}
p_{BS}(x, \tau; x') = \braket{x|e^{-\tau H}|x'} = \int_{\infty}^{\infty} \frac{dp}{2\pi}\braket{x|e^{-\tau H}|p}\braket{p|x'}
\end{equation}
Taking $p = i \frac{\partial}{\partial x}$, using $\braket{x|p} = e^{ipx}$:
\begin{equation}
p_{BS}(x, \tau;x') = e^{-r\tau}\int_{\infty}^{\infty}\frac{dp}{2\pi}exp\{- \frac{1}{2}\sigma^2p^2\tau + ip(x - x') + ip\tau(r - \frac{\sigma^2}{2}) \}
\end{equation}
Finally, perform the Gaussian integration:
\begin{equation}
p_{BS}(x, \tau;x') = e^{-r\tau} \frac{1}{\sqrt{2\pi \tau \sigma^2}} exp\{ - \frac{1}{2 \sigma^2 \tau}(x - x' + \tau(r - \frac{\sigma^2}{2}) )^2 \}
\end{equation}


\end{document}
