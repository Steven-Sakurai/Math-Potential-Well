%!TEX TS-program = xelatex  
%!TEX encoding = UTF-8 Unicode  
      
\documentclass[a4paper, 9pt]{article}
\usepackage[top=1.0in, bottom=1.0in, left=1.0in, right=1.0in]{geometry} 
\usepackage{indentfirst}        
\usepackage{float}
\usepackage{amsmath}
\usepackage{hyperref}
\usepackage{graphicx}

\title{ARMA-GARCH}
\author{SHIHENG SHEN}
\date{\today}
\begin{document}
\maketitle

\section{Temperature Model}
\subsection{Objective}
Build a robust time series model to forecast future temperature's interval. 


\subsection{Data Source}
HadCRUT4 is a gridded dataset of global historical surface temperature anomalies relative to a 1961-1990 reference period. Data are available for each month since January 1850, on a 5 degree grid. \url{http://www.metoffice.gov.uk/hadobs/hadcrut4/data/current/time_series/HadCRUT.4.5.0.0.monthly_ns_avg.txt}.

\subsection{Data adjustment}
From the shown url, grab the monthly global mean of temperature anomalies from 1850-2016, which amounts to a length of 2004. Using decompose() function in R to get the adjusted value.(See appendix)  

\subsection{Road Map}
We went through a lot of difficulties during the construction of a proper model. Below is the basic idea I've thinked about.\par
\begin{itemize}
\item Dealing with Statonarity: diff() or time trend?
\item Dealing with stochastic volatility: taking logarithm or exclude some data or both? Stochastic volatility model?
\item Choosing model: ARIMA or ARFIMA or ARMA-sGARCH or ARMA-eGARCH or ARFIMA-sGARCH or ARFIMA-eGARCH?
\end{itemize}







\newpage
\section{Appendix}

\subsection{R Code}
\subsubsection{Data Adjustment}
\begin{verbatim}
library(curl)
tmpf <- tempfile()
curl_download(url, tmpf)
gtemp <- read.table(tmpf)[, 1:2]
temp = gtemp$V2[1:2004]
library(TSA)
myTS = ts(as.numeric(temp), start = c(1850, 1), frequency = 12)
myTS.additive = decompose(myTS)
myTS.adjusted = myTS.additive$x - myTS.additive$seasonal
\end{verbatim}




\subsection{Figures}





\subsection{Details about \textit{decompose()} used in seasonal adjustment}
Type `decompose' in R console, we can see the source code of this function. The process of `type = additive' is listed below:
\begin{itemize}
\item the argument passed into decompose() is a `ts' object 
\item denote the argument ts(x, frequency = f). Create a filter using: filter = c(0.5, rep(1, f - 1), 0.5)/f.
\item trend = filter(x, filter)
\item season = x - trend, then compute f means of season with interval length f, the f means denoted by figure. Adjusting figure figure = figure - mean(figure)
\item seasonal is just length(x)/f times repetition of figure.
\item random = x - seasonal - trend
\end{itemize}


\end{document}