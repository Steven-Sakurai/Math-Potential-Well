%!TEX TS-program = xelatex  
%!TEX encoding = UTF-8 Unicode  
\documentclass[11pt,a4paper]{paper}
\usepackage{float}
\usepackage{indentfirst}
\usepackage{geometry} 
\usepackage{enumitem} 
\usepackage{amsmath}      
\usepackage{fontspec,xltxtra,xunicode}  

\usepackage[]{xeCJK}
\XeTeXlinebreaklocale “zh”  
\XeTeXlinebreakskip = 0pt plus 1pt minus 0.1pt 

\author{Steven Shen \quad \quad}
\title{Proposal}

\begin{document}
\maketitle

\section{Statement of the Problem}
\begin{enumerate}
	\item \textbf{What is the global temperature pattern in the long term?}
	\item \textbf{According to the pattern in $Q_1$, has there been periods with abnormal temperature?}
	\item \textbf{Does the temperature now rise abnormally compared with the past cyclical temperature rising?}  
	\item \textbf{How is carbon dioxide concentration correlated to global temperature?} 
	\item \textbf{Will there be a pause in global warming?} 
\end{enumerate}

\section{Review of Previous Work}
\begin{itemize}
	\item \textit{On the definition and identifiability of the alleged “hiatus” in global warming} by Stephan Lewandowsky, James S. Risbey \& Naomi Oreskes, 2015, https://www.nature.com/articles/srep16784 
	\item \textit{Global temperature change} by James Hansen, Makiko Sato, Reto Ruedy, Ken Lo, David W. Lea, and Martin Medina-Elizade, 2006, http://www.pnas.org/content/103/39/14288.abstract 
	\item \textit{Prospects for a prolonged slowdown in global warming in the early 21st century} by Thomas R. Knutson, Rong Zhang \& Larry W. Horowitz, 2016
	\item \textit{Interdecadal Oscillations and the Warming Trend in Global Temperature Time Series} by Vautard, Nature; London350.6316, 1991 
\end{itemize}


\section{Description of Data}
Data Resource:
\begin{enumerate}
	\item Global Temperature Data from NASA’s GISS data: http://data.giss.nasa.gov/gistemp/, whose land-ocean version combines land temperature observations with sea surface temperature data.

	\item UAH Data from satellite observations: http://vortex.nsstc.uah.edu/data/msu
 
\end{enumerate}



\end{document}