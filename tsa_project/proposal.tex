%!TEX TS-program = xelatex  
%!TEX encoding = UTF-8 Unicode  
\documentclass[11pt,a4paper]{paper}
\usepackage{float}
\usepackage{indentfirst}
\usepackage{geometry} 
\usepackage{enumitem} 
\usepackage{amsmath}      
\usepackage{fontspec,xltxtra,xunicode}  

\usepackage[]{xeCJK}
\XeTeXlinebreaklocale “zh”  
\XeTeXlinebreakskip = 0pt plus 1pt minus 0.1pt 

\author{Steven Shen \quad \quad}
\title{Proposal}

\begin{document}
\maketitle

\section{Statement of the Problem}
\textbf{Does the temperature Rise abnormally compared with the past cyclical temperature rising?}  \\
\textbf{Is CO2 the Main Cause?} \\
\textbf{Has there been a ‘pause’ in global warming?}\\


\begin{figure}[!htb]
\centering
\includegraphics[scale=.30]{pic2.png}
\end{figure}

\begin{figure}[!htb]
\centering
\includegraphics[scale=.40]{pic1.png}
\end{figure}

\section{Review of Previous Work}
There has been ferocious dabates on this problem. The major two are the `warmer' and the `lukewarmer'.
  \\
\indent Supporters of the `warmer’ and `lukewarmer’ views tend to favour different global temperature datasets, which show different temperature trends in recent years.  A favourite of the warmers is NASA’s GISS data, whose land-ocean version combines land temperature observations with sea surface temperature data. This data set was recently revised, with the new version showing a larger upward trend in temperature in recent years. The lukewarmers tend to favour the UAH data from satellite observations, also recently revised, with the new version showing a lower trend than before.

NASA states on https://climate.nasa.gov/evidence/:\\
Scientific evidence for warming of the climate system is unequivocal. \\
\indent  \quad - Intergovernmental Panel on Climate Change\\
\indent Yet I think the evidence is far from convincing. The selection of data is still under debates. Moreover measuring global temperature is not so simple.  Earth is a big place, with few observing stations, and every observing station is subject to biases from factors such as changes in the nature of its surroundings and in the time of day when observations are made. Measurements of temperature from space are indirect, and have potential biases from factors such as decaying satellite orbits.  All time series of global temperatures are therefore the result of complex processing of raw data, whose appropriateness can be questioned. \\






\section{Description of Data}
see ``./data/log.txt``

\end{document}